\documentclass[oneside]{article}			% document type

% SETTINGS
\usepackage{lmodern} 					% used family fonts
\usepackage[T1]{fontenc} 					% 8bit encoding

\usepackage[utf8]{inputenc} 				% accented letters
\usepackage[english]{babel} 				% right separation into distinct syllables

% UTILITY
\usepackage{listings}					% insert code
\lstset{breaklines=true}

\usepackage{longtable} 					% insert table
\usepackage{tabu}						% insert tabular
\usepackage{booktabs}

\usepackage{amsmath,amsfonts,amsthm} 	% insert math formulas
\usepackage{pgfplots}					% grafici
\pgfplotsset{compat=1.12}

\usepackage{graphicx}					% insert images
\usepackage{caption}					% insert special caption
\usepackage{subcaption}					% insert subcaption

\usepackage{geometry}					% insert page settings
\geometry{
	a4paper,
	top = 3.5cm,
	bottom = 3.5cm,
	left = 2cm,
	right = 2cm,
	heightrounded,
	headheight = 15pt,
	marginparwidth = 0pt,
	marginparsep = 0pt
}

\usepackage{placeins}					% insert barriers
\usepackage{exercise}
\renewcommand\ExerciseName{Question~}
\renewcommand\AnswerName{Answer to question}
\usepackage{lipsum}						% insert lipsum

\setcounter{section}{1}					% let's start from index 1
\renewcommand{\thesubsubsection}{\alph{subsubsection})}		% set letter for subsubsection

\newenvironment{adjustwidth}{\begin{center}\begin{tabular}{p{0.9\textwidth}}   }{\end{tabular} \end{center}}		% new environment, tabbed and centered (used in subsubsection)
\usepackage[ampersand]{easylist}
\usepackage{enumitem}			% used to make a list of items

% our test

%end our test

\begin{document}
	\title{Data Mining:\\Homework 1}
	\author{Simone Caldaro, Nana Zhu, Leonardo Martini}

	\maketitle

	\pagestyle{plain}
	\tableofcontents		% index of chapters
	\clearpage				% new page



	\addcontentsline{toc}{section}{Problem 1: shuffle a standard deck of cards}
	\section*{Problem 1}We shuffle a standard deck of cards, obtaining a permutation that is uniform over all 52! possible permutations.
	\subsection{Define a proper probability space $\Omega$ for the above random process}
	 	$\Omega$  is all the permutations of the 52 cards, that are 52! \newline
		 This corresponds to all the possible ways a deck can be shuffled.

	\subsection{Find the probability of the following events}

	% 1.2.a
	\begin{adjustwidth}
	\subsubsection{The first three cards include at least one ace}
	Let define the event $\mathit{E}$ = "The first three cards doesn't include any ace" \newline
	P($\mathit{E}$) = "the first card isn't an ace" $\times$ "the second card isn't an ace" $\times$ "the third card isn't an ace"
	\[ P(\bar{E}) = 1 - P(\mathit{E}) = 1 - \cfrac{48}{52} \times \cfrac{47}{51} \times \cfrac{46}{50}  = 0,217375566  \]
	that is the probability that at least one ace appear in the first three cards

	\end{adjustwidth}

	% 1.2.b
	\begin{adjustwidth}
	\subsubsection{The first five cards include exactly one ace}
	Exactly one ace in five cards can be seen as the sum of five events: $E_1$, $E_2$, $E_3$, $E_4$ and $E_5$, where $E_i$ corresponds to "the ace is in position $i$"
	\newline Each $E_i$ has the probability \[P(E_i) = \cfrac{4}{52} \times \cfrac{48}{51} \times \cfrac{47}{50} \times \cfrac{46}{49} \times \cfrac{44}{48} = 0,059894727 \]
	\newline This formula doesn't depend on the position of the ace, in fact the latter one corresponds to $P(E_1)$ and if we invert on the numerator the factor 4 with the factor 48 then the probability corresponds to $P(E_2)$.
	\[P(E) = P(E_1) + P(E_2) + P(E_3) + P(E_4) + P(E_5) = 5 \times P(E_i) = 5 \times 0,059894727 = 0,299473635 \]
	\end{adjustwidth}

	% 1.2.c
	\begin{adjustwidth}
	\subsubsection{The first three cards are all of the same rank (they are the same number or both are J, or all three are Q, etc.)}
	Whatever the first card is, the probability to have three cards of the same rank depends only by the second card and the third card.
	\newline $P(E)$ = "the $2^{nd}$ card is of the same rank of the $1^{st}$ one" $\times$ "the $3^{rd}$ card is of the same rank as the $1^{st}$ and $2^{nd}$ ones"
	\[P(E) = \cfrac{3}{51} \times \cfrac{2}{50} = 0,002352941 \]
	\end{adjustwidth}

	% 1.2.d
	\begin{adjustwidth}
	\subsubsection{The first five cards are all diamonds.}
	Everytime we pick a diamond card, the next step the number of diamond cards we can pick are decreased by one(at the beginning the number of diamond cards are 13).
	\[P(E) = \cfrac{13}{52} \times \cfrac{12}{51} \times \cfrac{11}{50} \times \cfrac{10}{49} \times \cfrac{9}{48} = 0,000495198\]
	\end{adjustwidth}

	% 1.2.e
	\begin{adjustwidth}
	\subsubsection{The first five cards form a full house (three of one rank and two of another rank).}
	A full house is composed by a pair plus a tris, that have different rank. \newline
	\\
	For example: $E$ = XXYYY is a full house, where:
	\begin{easylist}[itemize]
		& X can be whatever rank of a card;
		& Y must be a rank not equal to X.
	\end{easylist}
	\\		% empty line
	The full house event in the example is composed by these five events:
	\begin{easylist}[itemize]
	& "pick the first card (can be anyone)"
	& "pick the second card with the same rank of the first one in order to have a pair"
	& "pick the third card wiht rank different from the first two cards"
	& "pick the fouth card with the same rank as the third one, in order to have 2/3 of a tris"
	& "pick the fifth and last card in order to comple the tris"
	\end{easylist}
	\[P(E) = \cfrac{3}{51} \times \cfrac{48}{50} \times \cfrac{3}{49} \times \cfrac{2}{48} \times = 0,000144058\]
	This is the probability to have exactly this kind of full house (XXYYY). But we can also have:
	\begin{easylist}[itemize]
		& XYXYY
		& XYYXY
		& ...
	\end{easylist}
	\\
	If we count all these possible permutation, we'll end up with $\frac{5!}{3!2!}$ = 10 where 5 is the number of cards to permute, 3 and 2 are the rank of the cards that repeat in the permutation.
	\\\\
	The final probability to have a full house is: 10 $\times$ $P(E)$ = 0,00144058
	\end{adjustwidth}

	\subsection{(Optional) Develop some small programs in Python to perform simulations to check your answers} answer 3


	\clearpage				% new page
	% Problem 2
	\setcounter{section}{2}		% set number of section to 2
	\setcounter{subsection}{0}		% reset number of subsection
	\addcontentsline{toc}{section}{Problem 2: throw a set of 3 regular dice}
	\section*{Problem 2}
	You throw a set of 3 regular dice again and again, until for the first time you see a sum of 11 or a sum of 16.
	\subsection{Design an appropriate probability space for the above process}
	The sample space $\Omega$ is composed by a set of triple <$d_1$, $d_2$, $d_3$> , in which $d_i$ can be a number $\in [1, 6]$.
	\\
	For example, <1, 3, 4> $\in$ $\Omega$, while <6,2,8> $\notin$ $\Omega$
	\\\\
	The size of $\Omega$ is $6^3$ = 216
	\subsection{What is the probability that you stop because you see a sum of 16?}
	\begin{itemize}
		\item $E_1$ = "We get a sum of 11"
			\\
			The sum of the value of the three dices ($d_1$ + $d_2$ + $d_3$) can be 11 when:
			\begin{enumerate}[label=(\alph*)]
				\item 1 + 4 + 6 and there are 3! possible ways to have this result:
					\begin{enumerate}[label=\arabic*.]
					\item <1, 2, 6>
					\item <1, 6, 2>
					\item <2, 1, 6>
					\item <2, 6, 1>
					\item <6, 1, 2>
					\item <6, 2, 1>
					\end{enumerate}
				\item 1 + 5 + 5 and there are $\frac{3!}{2!}$ = 3 possible ways to have this result:
					\begin{enumerate}[label=\arabic*.]
						\item <1, 5, 5>
						\item <5, 1, 5>
						\item <1, 5, 5>
					\end{enumerate}
				\item 2 + 3 + 6 >> 3! ways
				\item 2 + 4 + 5 >> 3! ways
				\item 3 + 4 + 4 >> 3 ways
				\item 3 + 5 + 3 >> 3 ways
				\item 3 + 6 + 2 >> 3! ways.... but this triple has already been counted in (c)! So we can stop our count here, and don't include this row in the count!
			\end{enumerate}
			The probability to have as sum 11 is the number of events in which this sum is achieved (3 $\times$ 3! + 3 $\times$ 3 = 27 times) divided by all the number of all the possible events (216):
			\[P(E_1) = \frac{27}{216} = 0,125 \]
		\item $E_2$ = "We get a sum of 16"
			\\
			The sum of the value of the three dices ($d_1$ + $d_2$ + $d_3$) can be 16 when:
			\begin{enumerate}[label=(\alph*)]
				\item 4 + 6 + 6 >> 3 ways
				\item 5 + 5 + 6 >> 3 ways
			\end{enumerate}
			The probability to have as sum 16 is the number of events in which this sum is achieved (3 $\times$ 2 = 6 times) divided by all the number of all the possible events (216):
			\[P(E_2) = \frac{6}{216} = 0,02\overline{7}\]
		\item $E_s$ = "We stop rolling (we get either 11 or 16)" = $E_1$ $\cup$ $E_2$
		\[P(E_s) = P(E_1) + P(E_2) = \frac{27}{216} + \frac{6}{216} = \frac{33}{216} = 0,152\overline{7}\]
	\end{itemize}
	The probability to stop because a 16 is:
	\\
	\[ P(E_2|E_s) = \frac{P(E_2 \cap E_s)}{P(E_s)} = \frac{P(E_2)}{P(E_s)} = \frac{\frac{6}{216}}{\frac{33}{216}} = \frac{6}{33} = 0,\overline{18} \]


	\clearpage				% new page	
	% Problem 3
	\setcounter{section}{3}		% set number of section to 3
	\setcounter{subsection}{0}		% reset number of subsection
	\addcontentsline{toc}{section}{Problem 3: n men and m women sit in a round table}
	\section*{Problem 3} 
	A group of n men and m women go to a Chinese restaurant and sit in a round table, such that each person has to other person next to him/her
	\subsection{Describe a sample space that describes the random process}
	answer 1		% TODO answer here
	\subsection{Find the expected number of men who will be sitted next to at least one woman}
	answer 2		% TODO answer here
		
		
	\clearpage				% new page
	% Problem 4
	\setcounter{section}{4}		% set number of section to 4
	\setcounter{subsection}{0}		% reset number of subsection
	\addcontentsline{toc}{section}{Problem 4: search engine for recipes}
	\section*{Problem 4} 


\end{document}
